%TODO: Add solutions at the end of the document
%TODO: Check spacing

\documentclass[12pt]{article}
\usepackage[margin=1in]{geometry}
\usepackage{amsmath}

\setlength{\parindent}{0pt}
\setlength{\parskip}{8pt}

\title{Week 7: Systems Applications \& Inequalities}
\author{
	Student: SA\\
	Tutor: Rachel Eglash}
\date{October 22, 2025}

\begin{document}
	
	\maketitle

	\section*{Session 7.1 \\ Systems Word Problems Mixture, Motion, Break-Even}
	
	    \newpage
	
	\section*{Quick Reference: Problem Type Formulas}
	
	    \subsection*{Mixture Problems}
	
		    \textbf{Key Formula:}
			(concentration$_1$)(volume$_1$) + (concentration$_2$)(volume$_2$) = (concentration$_{\text{final}}$)(volume$_{\text{final}}$)
		
		    \textbf{Two Equations Needed:}
		    \begin{enumerate}
				\item Total volume equation
		    	\item Total substance (acid, alcohol, etc.) equation
		    \end{enumerate}
	
	    \subsection*{Motion Problems}
	
		    \textbf{Key Formula:}
			distance = rate $\times$ time \quad ($d = rt$)
		
		    \textbf{Two Equations Needed:}
			\begin{enumerate}
				\item Total distance equation (add if opposite directions, subtract if same direction)
				\item Distance equations for each object: $d_1 = r_1 t$ and $d_2 = r_2 t$
			\end{enumerate}
	
	    \subsection*{Break-Even Problems}
	
		    \textbf{Key Formula:}
			Total Cost = Total Revenue
		
		    \textbf{Two Equations Needed:}
			\begin{enumerate}
				\item Cost: $C = \text{Fixed Costs} + (\text{variable cost per item})(x)$
				\item Revenue: $R = (\text{price per item})(x)$
				\item Set them equal: $C = R$
			\end{enumerate}
	
	        \newpage
	
	\section*{Homework 7.1: Systems Word Problems}
	
	    \subsection*{Instructions}:
	        For each problem,
	        \begin{enumerate}
                \item Define your variables with units
                \item Write your system of equations
                \item Solve using any method
                \item Check your answer for reasonableness
            \end{enumerate}
            
	        \newpage
	
        \subsection*{Homework Problem 1: Mixture Problem}
            
            A pharmacist needs to prepare 200 mL of a 15 mg/mL alcohol solution.\\\\
            She has solution "A", which is a 10 mg/mL solution.\\\\
            She has solution "B", which is a 25 mg/mL solution.\\\\
            How much of each should she mix?\\
        
            \textbf{Variables:}\\
            $A$ = number of mL of 10 mg/mL solution\\
            $B$ = number of mL of 25 mg/mL solution\\
        
            \textbf{Understanding Check:}\\\\
            Total alcohol needed = $200$ mL $\times$ $15$ mg/mL = \underline{\hspace{1in}} mg\\
        
            \textbf{System:}\\\\
            1. Equation for total mL: \underline{\hspace{4in}}\\\\
            2. Expression for mg from solution A: \underline{\hspace{3.5in}}\\\\
            3. Expression for mg from solution B: \underline{\hspace{3.5in}}\\\\
            4. Equation for total mg: \underline{\hspace{4in}}\\
        
            \textbf{Solution:}\\\\
            $A$ = \underline{\hspace{1in}} mL\\\\
            $B$ = \underline{\hspace{1in}} mL\\
        
            \textbf{Check Your Answer:}\\\\
            Does $A + B = 200$? \underline{\hspace{3in}}\\\\
            Does the total alcohol equal 3000 mg? \underline{\hspace{4.5in}}
        
            \newpage
        
        \subsection*{Homework Problem 2: Motion Problem}

            Two trains leave the same station at the same time.\\\\
            They travel in opposite directions.\\\\
            Train A travels at 75 mph.\\\\
            Train B travels at 65 mph.\\\\
            After how many hours will they be 420 miles apart?\\
        
            \textbf{Variables:}\\
            $t$ = time in hours\\
            $A$ = distance traveled by train A in miles\\
            $B$ = distance traveled by train B in miles\\
        
            \textbf{Understanding Check:}\\\\
            Combined speed = \underline{\hspace{1in}} + \underline{\hspace{1in}} = \underline{\hspace{1in}} mph\\
        
            \textbf{System:}\\\\
            1. Equation for total distance: \underline{\hspace{4in}}\\\\	
            2. Equation for train A distance: \underline{\hspace{4in}}\\\\
            3. Equation for train B distance: \underline{\hspace{4in}}\\
        
            \textbf{Solution:}\\\\
            $t$ = \underline{\hspace{1in}} hours\\
        
            \textbf{Check Your Answer:}\\\\
            Train A distance: \underline{\hspace{2in}} miles\\\\
            Train B distance: \underline{\hspace{2in}} miles\\\\
            Total: \underline{\hspace{2in}} miles (should equal 420)
        
            \newpage
        
        \subsection*{Homework Problem 3: Break-Even Problem}
        
            A bakery makes cakes.\\\\	
            Fixed costs are \$1500 per month,\\
            and each cake costs \$20 to produce.\\\\
            Each cake sells for \$45.\\\\
            How many cakes must be sold to break even?\\
        
            \textbf{Variables:}\\
            $x$ = number of cakes\\
            $C$ = total cost in dollars\\
            $R$ = total revenue in dollars\\
        
            \textbf{Understanding Check:}\\\\
            Profit per cake = \underline{\hspace{1in}} - \underline{\hspace{1in}} = \underline{\hspace{1in}}\\
        
            \textbf{System:}\\\\
            1. Equation for total cost: $C$ = \underline{\hspace{2in}}\\\\	
            2. Equation for total revenue: $R$ = \underline{\hspace{2in}}\\\\
            3. Break even means: \underline{\hspace{0.3in} $C = R$ \hspace{0.3in}}\\
        
            \textbf{Solution:}\\\\
            $x$ = \underline{\hspace{1in}} cakes\\

            \textbf{Check Your Answer:}\\\\
            Total cost: \underline{\hspace{5in}}\\\\
            Total revenue: \underline{\hspace{2in}}\\\\
            Are they equal? \underline{\hspace{1in}}
        
            \newpage
        
        \subsection*{Homework Problem 4: Mixture Problem Challenge}

            A coffee shop mixes two types of beans.\\\\
            Premium beans cost \$12 per pound.\\
            Regular beans cost \$8 per pound.\\\\
            The shop wants to make 50 pounds of a blend.\\\\
            The blend should cost \$9.60 per pound.\\\\
            How many pounds of each type should be used?\\
        
            \textbf{Variables:}\\
            $p$ = pounds of premium beans\\
            $r$ = pounds of regular beans\\
        
            \textbf{Understanding Check:}\\
            Total cost of blend = $50$ pounds $\times$ \$$9.60$ per pound = \underline{\hspace{1in}}\\
            This problem is like a mixture problem, but with \textbf{cost} instead of concentration!\\
        
            \textbf{System:}\\
            1. Equation for total pounds: \underline{\hspace{4in}}\\\\
            2. Expression for cost from premium beans: \underline{\hspace{3.5in}}\\\\
            3. Expression for cost from regular beans: \underline{\hspace{3.5in}}\\\\
            4. Equation for total cost: \underline{\hspace{4in}}\\
        
            \textbf{Solution:}\\\\
            $p$ = \underline{\hspace{1in}} pounds\\\\
            $r$ = \underline{\hspace{1in}} pounds\\
        
            \textbf{Check Your Answer:}\\
            Does $p + r = 50$? \underline{\hspace{2in}}\\\\
            Does the total cost equal \$480? \underline{\hspace{3in}}\\\\
            Makes sense: Should use more regular beans (cheaper) than premium beans? \underline{\hspace{2.5in}}
        
            \newpage
	
\end{document}