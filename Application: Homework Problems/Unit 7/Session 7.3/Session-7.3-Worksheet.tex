%TODO: Add solutions at the end of the document
%TODO: Add graphs where needed
%TODO: Check spacing after adding graphs

\documentclass[12pt]{article}
\usepackage[margin=1in]{geometry}
\usepackage{amsmath}

\setlength{\parindent}{0pt}
\setlength{\parskip}{8pt}

\title{Week 7: Systems Applications \& Inequalities}
\author{
	Student: SA\\
	Tutor: Rachel Eglash}
\date{October 24, 2025}

\begin{document}
	
	\maketitle

	\section*{Session 7.3 \\ Linear Modeling \& Fit-by-Eye}

		\newpage

	\section*{Quick Reference: Linear Modeling}

		\subsection*{What is a Line of Best Fit?}

			A \textbf{line of best fit} (or trend line) is a straight line that best represents the data on a scatter plot.

			\textbf{Key Ideas:}
			\begin{itemize}
				\item Not all points will be exactly on the line
				\item The line shows the general trend or pattern
				\item We use the line to make predictions
				\item The line minimizes the distance from all the data points
			\end{itemize}

		\subsection*{Steps to Create a Line of Best Fit}

			\textbf{1. Plot the Data}
			\begin{itemize}
				\item Create a scatter plot of all data points
				\item Label axes with variable names and units
				\item Choose an appropriate scale
			\end{itemize}

			\textbf{2. Draw the Line}
			\begin{itemize}
				\item Use a ruler or straightedge
				\item Try to balance points above and below the line
				\item The line should follow the general trend
				\item Extend the line across the entire graph
			\end{itemize}

			\textbf{3. Find the Equation}
			\begin{itemize}
				\item Pick two points ON your line (they don't have to be data points)
				\item Calculate the slope: $m = \frac{y_2 - y_1}{x_2 - x_1}$
				\item Find the y-intercept using point-slope form
				\item Write in slope-intercept form: $y = mx + b$
			\end{itemize}

			\newpage

		\subsection*{Residuals}

			A \textbf{residual} measures how far each data point is from the line of best fit.

			\textbf{Formula:}
			$$\text{Residual} = \text{Actual value} - \text{Predicted value}$$

			\textbf{Interpretation:}
			\begin{itemize}
				\item \textbf{Positive residual}: data point is above the line
				\item \textbf{Negative residual}: data point is below the line
				\item \textbf{Zero residual}: data point is exactly on the line
				\item Smaller residuals mean better fit
			\end{itemize}

		\subsection*{Interpreting Slope and y-Intercept}

			\textbf{Slope ($m$):}
			\begin{itemize}
				\item Rate of change
				\item How much $y$ changes for each 1-unit increase in $x$
				\item Include units: ``For every [unit of $x$], [variable $y$] changes by [slope] [unit of $y$]''
			\end{itemize}

			\textbf{y-Intercept ($b$):}
			\begin{itemize}
				\item Starting value when $x = 0$
				\item May or may not make sense in context
				\item Include units: ``When [variable $x$] is 0, [variable $y$] is [y-intercept] [unit of $y$]''
			\end{itemize}

			\newpage

	\section*{Worksheet 7.3: Linear Modeling}

		\subsection*{Instructions}:
			
			For each problem,
			\begin{enumerate}
				\item Plot the data points on graph paper
				\item Draw a line of best fit by eye
				\item Find the equation using two points on your line
				\item Calculate residuals for specified data points
				\item Interpret the slope and y-intercept in context
			\end{enumerate}

			\newpage

		\subsection*{Worksheet Problem 1: Temperature and Hot Chocolate Sales}

			A café tracks hot chocolate sales at different temperatures.

			\begin{center}
				\begin{tabular}{|c|c|}
					\hline
					Temperature ($^\circ$F) & Hot Chocolates Sold \\
					\hline
					30 & 45 \\
					35 & 42 \\
					40 & 38 \\
					45 & 35 \\
					50 & 30 \\
					55 & 28 \\
					\hline
				\end{tabular}
			\end{center}

			\subsubsection*{Part A}

				Choose your scale:
				\begin{itemize}
					\item x-axis: Temperature from \underline{\hspace{1in}} to \underline{\hspace{1in}} (include units)
					\item y-axis: Sales from \underline{\hspace{1in}} to \underline{\hspace{1in}} (include units)
				\end{itemize}

				Add your scale and label axes.\\\\
				Plot the points.

				\vspace{8cm}

			\subsubsection*{Part B}
				
				Draw a line of best fit (using a different color).

				\newpage

			\subsubsection*{Part C}
				
				Choose two points ON your line and find the equation.

				Point 1: $(\underline{\hspace{1in}}, \underline{\hspace{1in}})$

				Point 2: $(\underline{\hspace{1in}}, \underline{\hspace{1in}})$

				Calculate the slope:

				\vspace{2cm}

				Find the y-intercept using point-slope form:

				\vspace{2cm}

				Equation: \underline{\hspace{3in}}

			\subsubsection*{Part D}
			
				Calculate residuals for $x = 30$, $x = 40$, and $x = 50$:

				For $x = 30$:
				\begin{itemize}
					\item Actual value: \underline{\hspace{1in}}
					\item Predicted value (from equation): \underline{\hspace{1in}}
					\item Residual = Actual $-$ Predicted = \underline{\hspace{1in}}
				\end{itemize}

				For $x = 40$:
				\begin{itemize}
					\item Actual value: \underline{\hspace{1in}}
					\item Predicted value: \underline{\hspace{1in}}
					\item Residual = \underline{\hspace{1in}}
				\end{itemize}

				For $x = 50$:
				\begin{itemize}
					\item Actual value: \underline{\hspace{1in}}
					\item Predicted value: \underline{\hspace{1in}}
					\item Residual = \underline{\hspace{1in}}
				\end{itemize}

				\newpage

			\subsubsection*{Part E}
			
				Interpret the slope in context:

				For every \underline{\hspace{2in}} (include units) increase in temperature,

				the number of hot chocolates sold
				
				\underline{\hspace{2in}} (increases/decreases) by about \underline{\hspace{1.4in}} (include units).

			\subsubsection*{Part F}
			
				Interpret the y-intercept in context:

				When the temperature is $0^\circ$F, the model predicts \underline{\hspace{1in}} hot chocolates sold.

			\subsubsection*{Part G}
			
				Does the y-intercept make sense in this context? Why or why not?

				\vspace{3cm}

			\subsubsection*{Part H}
			
				Use your model to predict sales at $60^\circ$F:

				\vspace{2cm}

				Is this prediction reasonable? Explain.

				\vspace{2cm}

				\newpage

		\subsection*{Worksheet Problem 2: Study Time and Quiz Score}

			A teacher tracks student study time and quiz scores.

			\begin{center}
				\begin{tabular}{|c|c|}
					\hline
					Study Time (hours) & Quiz Score (\%) \\
					\hline
					0.25 & 67 \\
					0.5 & 70 \\
					0.75 & 73 \\
					1.0 & 79 \\
					1.25 & 88 \\
					1.5 & 93 \\
					\hline
				\end{tabular}
			\end{center}

			\subsubsection*{Part A}
			
				Add a scale and label axes.\\\\
				Plot the points.

				\vspace{8cm}

			\subsubsection*{Part B}

				Draw a line of best fit (using a different color).

				\newpage

			\subsubsection*{Part C}

				Find the equation of the line of best fit using two points on your line.

				\vspace{5cm}

			\subsubsection*{Part D}
				
				Calculate residuals for $x = 0.5$, $x = 1.0$, and $x = 1.5$:

				\vspace{5cm}

			\subsubsection*{Part E}
				
				Use your model to predict the quiz score for someone who studies 4 hours:

				\vspace{2cm}

			\subsubsection*{Part F}
			
				Is this prediction reasonable? Why or why not?

				(Hint: Can quiz scores go above 100\%?)

				\vspace{2cm}

				\newpage

		\subsection*{Worksheet Problem 3: Car Value Over Time}

			A used car dealer tracks how a car's value changes with age.

			\begin{center}
				\begin{tabular}{|c|c|}
					\hline
					Car Age (years) & Value (\$1000s) \\
					\hline
					1 & 23 \\
					2 & 20 \\
					3 & 17 \\
					4 & 15 \\
					5 & 12 \\
					6 & 9 \\
					\hline
				\end{tabular}
			\end{center}

			\subsubsection*{Part A}
			
				Add a scale and label axes.\\\\
				Plot the points.

				\vspace{8cm}

			\subsubsection*{Part B}
			
				Draw a line of best fit (using a different color).

				\newpage

			\subsubsection*{Part C}
			
				Find the equation of the line of best fit.

				\vspace{4cm}

			\subsubsection*{Part D}
			
				What does the y-intercept represent?

				\vspace{2cm}

				Does this make sense? (What was the car worth when new?)

				\vspace{2cm}

			\subsubsection*{Part E}
			
				According to your model, when will the car be worth \$0?

				\vspace{3cm}

			\subsubsection*{Part F}
			
				Is this prediction realistic? Why or why not?

				\vspace{3cm}

				\newpage

		\subsection*{Worksheet Problem 4: Plant Growth}

			A biology student measures plant height over time.

			\begin{center}
				\begin{tabular}{|c|c|}
					\hline
					Days & Height (cm) \\
					\hline
					0 & 3 \\
					5 & 5 \\
					10 & 9 \\
					15 & 12 \\
					20 & 15 \\
					25 & 17 \\
					\hline
				\end{tabular}
			\end{center}

			\subsubsection*{Part A}
			
				Add a scale and label axes.\\\\
				Plot the points.\\\\
				Draw a line of best fit (using a different color).

				\vspace{8cm}

			\subsubsection*{Part B}
			
				Find the equation of the line of best fit.

				\newpage

			\subsubsection*{Part C}
			
				Interpret the y-intercept:

				\vspace{2cm}

			\subsubsection*{Part D}
			
				Calculate the residual for day 10:

				\vspace{3cm}

			\subsubsection*{Part E}
			
				Predict the height after 30 days:

				\vspace{2cm}

			\subsubsection*{Part F}
			
				If the plant can only grow to a maximum of 25 cm,\\
				when will it reach this height according to your model?

\end{document}