%TODO: Add solutions at the end of the document
%TODO: Add graphs where needed
%TODO: Check spacing after adding graphs

\documentclass[12pt]{article}
\usepackage[margin=1in]{geometry}
\usepackage{amsmath}

\setlength{\parindent}{0pt}
\setlength{\parskip}{8pt}

\title{Week 7: Systems Applications \& Inequalities}
\author{
	Student: SA\\
	Tutor: Rachel Eglash}
\date{October 23, 2025}

\begin{document}
	
	\maketitle
	
	\section*{Session 7.2 \\ Systems of Inequalities}
	
	    \newpage
	
	\section*{Quick Reference: Graphing Inequalities}
	
	    \subsection*{Single Linear Inequality}
	
	        \textbf{Steps to Graph:}
	        \begin{enumerate}
	    	    \item Write in slope-intercept form if possible: $y = mx + b$
	    	    \item Graph the boundary line:
	    	    \begin{itemize}
	    		    \item Use \textbf{solid line} for $\leq$ or $\geq$ (boundary included)
		    	    \item Use \textbf{dashed line} for $<$ or $>$ (boundary not included)
	    	    \end{itemize}
	    	    \item Choose a test point (often $(0,0)$ if not on the line)
		        \item Shade the region where the inequality is true
	        \end{enumerate}
	
	    \subsection*{System of Linear Inequalities}
	
	        \textbf{Steps to Graph:}
    	    \begin{enumerate}
	            \item Graph each inequality on the same coordinate plane
		        \item The \textbf{solution region} is where ALL shaded areas overlap
	    	    \item The solution region is called the \textbf{feasible region}
	        	\item Any point in this region satisfies all inequalities
	        \end{enumerate}
	
	        \textbf{Key Vocabulary:}
	        \begin{itemize}
		        \item \textbf{Feasible region}: the solution set (overlapping shaded area)
		        \item \textbf{Vertex}: corner point where boundary lines intersect
		        \item \textbf{Bounded}: the feasible region is enclosed (finite area)
		        \item \textbf{Unbounded}: the feasible region extends infinitely
	        \end{itemize}
	
	        \newpage
	
	\section*{Homework 7.2: Systems of Inequalities}
	
	    \subsection*{Instructions}:
		
	        For each problem,
	        \begin{enumerate}
		        \item Graph each inequality carefully (solid vs. dashed lines)
		        \item Identify and shade the feasible region
		        \item Verify with a test point
		        \item Answer interpretation questions
	        \end{enumerate}
	
	        \newpage
	
	    \subsection*{Homework Problem 1: Basic Inequalities}
		
            \textbf{Graph the inequality:}\\\\
			$y > \frac{3}{4} x - 1$
			\begin{itemize}
				\item Slope: \underline{\hspace{1in}}
				\item y-intercept: \underline{\hspace{1in}}
				\item Solid or dashed line? \underline{\hspace{2in}}
				\item Shade above or below the line? \underline{\hspace{2in}}
			\end{itemize}
				
			\vspace{4cm}
		
		    \textbf{Graph the inequality:}\\\\
			$y \leq -2x + 1$
			\begin{itemize}
				\item Slope: \underline{\hspace{1in}}
	    		\item y-intercept: \underline{\hspace{1in}}
				\item Solid or dashed line? \underline{\hspace{2in}}
				\item Shade above or below the line? \underline{\hspace{2in}}
			\end{itemize}
				
        	\newpage
	
    	\subsection*{Homework Problem 2: Basic System}
		
            \textbf{Graph the system of inequalities:}\\\\
			$y > 2x - 3$\\
			$y \leq -x + 4$\\
	
	        \newpage
	
	    \subsection*{Homework Problem 3: Bounded Inequalities}
		
            \textbf{Graph the compound inequality:}\\\\
			$0 \leq x \leq 5$\\
			
		    \vspace{4.5cm}
			
		    \textbf{Graph the compound inequality:}\\\\
			$0 \leq y \leq 4$\\
			
		    \vspace{4.5cm}
		
		    \textbf{Graph the system of inequalities:}\\\\
			$0 \leq x \leq 5$\\
			$0 \leq y \leq 4$\\
			
	        \newpage
	
	        \textbf{Understanding Check:}\\\\
		    What shape will the feasible region be? \underline{\hspace{2in}}\\\\
		    Is it bounded or unbounded? \underline{\hspace{2in}}\\
	
        	\textbf{Vertices of Feasible Region:}\\\\
		    List all four corner points:\\\\
			\hspace{1cm} \underline{\hspace{6in}}\\\\
	
	        \textbf{Check:}\\
		    Pick a test point in the feasible region and show that it satisfies both inequalities.
		
		    \vspace{3cm}
		
		    Pick a test point NOT in the feasible region\\
			and show that it does NOT satisfy at least one of the inequalities.\\
	
	        \newpage
	
	    \subsection*{Homework Problem 4: Application - Manufacturing}
		
            \textbf{A factory makes tables and chairs.\\\\
		    Each table requires 4 hours of labor.\\
		    Each chair requires 2 hours of labor.\\\\
		    The factory has \underline{at most} 20 hours of labor per day.\\\\
		    They must make \underline{at least} 2 tables per day.\\
		    They can make at most 6 chairs per day.\\\\
		    Write and graph a system of inequalities.}\\
	
	        \textbf{Variables:}\\
	        $t$ = number of tables\\
	        $c$ = number of chairs\\
	
	        \textbf{System of Inequalities:}\\\\
	        1. Labor constraint: \underline{\hspace{4in}}\\\\
	        2. Minimum tables: \underline{\hspace{4in}}\\\\
	        3. Maximum chairs: \underline{\hspace{4in}}\\\\
	        4. Non-negative: $t \geq 0$ and $c \geq 0$ (already implied by context)\\
	
	        \newpage
	
	        \textbf{Interpretation Questions:}\\\\
		    Can they make 3 tables and 4 chairs?\\\\
			Why or why not?
	
		    \vspace{4cm}
		
		    Can they make 4 tables and 2 chairs?\\\\
			Why or why not?
			
		    \vspace{4cm}
	
		    What is the maximum number of chairs if they make exactly 2 tables?
	
\end{document}