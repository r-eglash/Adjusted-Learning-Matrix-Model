\documentclass{article}
\usepackage{amsmath}
\usepackage{amssymb}
\usepackage{geometry}

\geometry{margin=1in}

\title{Math 2BL Project Proposal Rough Draft}
\author{Rachel Eglash}
\date{October 6, 2025}

\begin{document}

\maketitle

\section{Project Name}
    Adjusted Learning Matrix Model

\section{Project Description}
    The project will create a graphical user interface (GUI) designed for an instructor (tutor or teacher) who is looking to create an adjustable learning plan for an individual student. This program is focused on providing equitable, accessible, individualized learning plans for students with learning differences, although this program is made in the Universal Design for Learning (UDL) style to benefit all students. The program will use linear algebra and matrices, with an instructor input matrix, a transformation matrix, and an adapted instruction output matrix to provide guidance to the instructor.

    \subsection{Overview of Matrices}

        \[
            y = Ax
        \]

        \textbf{Input Matrix:}
        \[
            x = \begin{bmatrix}
            \frac{h-5}{2} \\
            \frac{c-5}{2} \\
            \frac{p-5}{2} \\
            \frac{m-5}{2} \\
            \end{bmatrix}
        \]

       where:
            \begin{itemize}
               \item $h$ = independence/help provided ($0 = $ no independence, $10 = $ full independence)
               \item $c$ = observed student confidence ($0 = $ no confidence, $10 = $ full confidence)
               \item $p$ = persistence ($0 = $ gave up immediately, $10 = $ never gave up)
                \item $m$ = accuracy \& mistake severity/type ($0 = $ full conceptual error, $10 = $ no errors)
           \end{itemize}

        \textbf{Transformation Matrix:}
            \[
                A = \begin{bmatrix} 
                a_{11} & a_{12} & a_{13} & a_{14} \\
                a_{21} & a_{22} & a_{23} & a_{24} \\
                a_{31} & a_{32} & a_{33} & a_{34}
                \end{bmatrix}
                \text{*need to determine matrix values}
            \]

        \textbf{Output Matrix:}
            \[
                y = \begin{bmatrix} d \\ s \\ t \end{bmatrix}
            \]

        where:
            \begin{itemize}
                \item $d$ = adjusted difficulty level, $[-1, 1]$ ($- = $ decrease, $0 = $ same, $+ = $ increase)
                \item $s$ = adjusted support level, $[-1, 1]$ ($- = $ increase, $0 = $ same, $+ = $ decrease)
                \item $t$ = adjusted time/pacing, $[-1, 1]$ ($- = $ slow down, $0 = $ same, $+ = $ speed up)
            \end{itemize}

    \subsection{Instructor Input Matrix}

        The instructor will be given prompts to enter in values based on easily measurable observations on how the student did on that assignment. For example, on the user (instructor) end, this could look like:

        \begin{itemize}
            \item Enter student's independence/amount of help provided (0=no independence, 10=full independence):
            \item Enter observed student confidence (0=no confidence, 10=full confidence):
            \item Enter student's persistence (0=gave up immediately, 10=never gave up):
            \item Enter student's accuracy \& mistake severity/type* (0=full conceptual error, 10=no errors):
        \end{itemize}

        On the programming end, this corresponds to a matrix $x$:

        \[
            x = \begin{bmatrix}
                \frac{h-5}{2} \\
                \frac{c-5}{2} \\
                \frac{p-5}{2} \\
                \frac{m-5}{2} \\
            \end{bmatrix}
        \]

        where the variables $h, c, p,$ and $m$ range from $0$ to $10$ (inclusive) and correspond to the values entered in by the user (instructor), where $h = $ independence/help, $c = $ confidence, $p = $ persistence, and $m = $ accuracy \& mistake severity/type. These values are then normalized so that the values in the matrix $x$ range from $-1$ to $1$ (inclusive).

        \textbf{Note:} Need to refine the ``mistake severity/type'' values to account for things like arithmetic errors (minor), forgetting units (minor), incorrect units (moderate), minor conceptual error (moderate), full conceptual error (severe).

    \subsection{Adapted Instruction Output Matrix}

        Once the instructor has entered in the values for matrix $x$, the program will output recommendations on how the difficulty level, support level, and pace of material should be adjusted. For example, on the user (instructor) end, this could look like:

        \begin{itemize}
            \item The difficulty level should: increase/decrease slightly/moderately/significantly OR stay the same
            \item The support provided should: increase/decrease slightly/moderately/significantly OR stay the same
            \item The pacing of the material should: speed up/slow down slightly/moderately/significantly OR stay the same
        \end{itemize}

        On the programming end, this corresponds to a matrix $y$:

        \[
            y = \begin{bmatrix} d \\ s \\ t \end{bmatrix}
        \]

        where $d, s, t$ range from $-1$ to $1$, and guide the recommendations given to the user (instructor).

        \newpage

    \subsubsection{Pseudocode for Output Matrix}

        \begin{verbatim}
            Print "The difficulty level should "
            If d = 0, print "stay the same."
            Else, if d < 0, print "decrease "
                If d <= -0.7, print "significantly."
                Else, if d < -0.3, print "moderately."
                Else, print "slightly."
            Else, if d > 0, print "increase "
                If d >= 0.7, print "significantly."
                Else, if d > 0.3, print "moderately."
                Else, print "slightly."
            Else, error

            Print "The support provided should "
            If s = 0, print "stay the same."
            Else, if s < 0, print "increase "
                If s <= -0.7, print "significantly."
                Else, if s < -0.3, print "moderately."
                Else, print "slightly."
            Else, if s > 0, print "decrease "
                If s >= 0.7, print "significantly."
                Else, if s > 0.3, print "moderately."
                Else, print "slightly."
            Else, error

            Print "The pacing should "
            If t = 0, print "stay the same."
            Else, if t < 0, print "slow down "
                If t <= -0.7, print "significantly."
                Else, if t < -0.3, print "moderately."
                Else, print "slightly."
            Else, if t > 0, print "speed up "
                If t >= 0.7, print "significantly."
                Else, if t > 0.3, print "moderately."
                Else, print "slightly."
            Else, error
        \end{verbatim}

        \newpage

    \subsection{Transformation Matrix}

        This matrix is only visible on the programmer's end. This matrix will include values that represent how each input from the instructor affects the output for how the instruction should be adapted.

        On the programming end, this corresponds to a matrix $A$:

        \[
            A = \begin{bmatrix} 
                a_{11} & a_{12} & a_{13} & a_{14} \\
                a_{21} & a_{22} & a_{23} & a_{24} \\
                a_{31} & a_{32} & a_{33} & a_{34}
            \end{bmatrix}
            \text{**(need to determine what these values are)}
        \]

        where values in the matrix range from $-1$ to $1$.

        For example, if a lot of help was given by the instructor meaning the student had no independence ($h$ is close to $0$), then this would correspond to decreasing the difficulty (negative $d$), increasing the support provided (negative $s$), and/or slowing down the pace (negative $t$).

        A more specific example is if the instructor observes that the student's confidence is low ($c$ close to $0$) even though the student is getting the correct answer ($m=10$), then this would correspond to increasing the support provided (negative $s$), but keeping the same difficulty ($d=0$) and pace ($t=0$), in order to build up the student's confidence.

        \textbf{Note:} Need to determine the values in the matrix. Also, need to think about if output matrix $y$ represents adjusted values (change) or an absolute value.

    \subsubsection{Matrix Equations}

        \[
            \begin{bmatrix} d \\ s \\ t \end{bmatrix}
            = 
            \begin{bmatrix} 
                a_{11} & a_{12} & a_{13} & a_{14} \\
                a_{21} & a_{22} & a_{23} & a_{24} \\
                a_{31} & a_{32} & a_{33} & a_{34}
            \end{bmatrix}
            \begin{bmatrix}
                \frac{h-5}{2} \\
                \frac{c-5}{2} \\
                \frac{p-5}{2} \\
                \frac{m-5}{2} \\
            \end{bmatrix}
            =
            \frac{1}{5}
            \begin{bmatrix}
                a_{11}(h - 5) + a_{12}(c - 5) + a_{13}(p - 5) + a_{14}(m - 5) \\
                a_{21}(h - 5) + a_{22}(c - 5) + a_{23}(p - 5) + a_{24}(m - 5) \\
                a_{31}(h - 5) + a_{32}(c - 5) + a_{33}(p - 5) + a_{34}(m - 5)
            \end{bmatrix}
        \]

        Expanding:

        \[
            \left\{
                \begin{aligned}
                    d &= a_{11}(h - 5) + a_{12}(c - 5) + a_{13}(p - 5) + a_{14}(m - 5) \\
                    s &= a_{21}(h - 5) + a_{22}(c - 5) + a_{23}(p - 5) + a_{24}(m - 5) \\
                    t &= a_{31}(h - 5) + a_{32}(c - 5) + a_{33}(p - 5) + a_{34}(m - 5)
                \end{aligned}
            \right.
        \]

    \subsubsection{Test to Determine Starting Values for Matrix $A$}

        \textbf{Given:}
        \begin{itemize}
            \item Full independence: $h = 10$
            \item Full confidence: $c = 10$
            \item Never gave up: $p = 10$
            \item No errors: $m = 10$
        \end{itemize}

        \textbf{Then:}
        \begin{itemize}
            \item Difficulty increases significantly: $d = 1$
            \item Support provided decreases significantly: $s = 1$
            \item Pacing speeds up significantly: $t = 1$
        \end{itemize}

        \[
            \begin{bmatrix} 
                a_{11} & a_{12} & a_{13} & a_{14} \\
                a_{21} & a_{22} & a_{23} & a_{24} \\
                a_{31} & a_{32} & a_{33} & a_{34}
            \end{bmatrix}
            \begin{bmatrix} 1 \\ 1 \\ 1 \\ 1 \end{bmatrix}
            = 
            \begin{bmatrix}
                a_{11} + a_{12} + a_{13} + a_{14} \\
                a_{21} + a_{22} + a_{23} + a_{24} \\
                a_{31} + a_{32} + a_{33} + a_{34}
            \end{bmatrix}
            =
            \begin{bmatrix} 1 \\ 1 \\ 1 \end{bmatrix}
        \]

        This yields:

        \[
            \left\{
                \begin{aligned}
                    a_{11} + a_{12} + a_{13} + a_{14} &= 1 \\
                    a_{21} + a_{22} + a_{23} + a_{24} &= 1 \\
                    a_{31} + a_{32} + a_{33} + a_{34} &= 1
                \end{aligned}
            \right.
        \]

        Since the range of possible values for $h, c, p,$ and $m$ is $[0, 10]$, the range of possible values for each entry in matrix $x$ is $[-1, 1]$.

        Since the range of possible values for $d, s,$ and $t$ is $[-1, 1]$, the range of possible values for each entry in matrix $y$ is $[-1, 1]$.

        Therefore, if each entry in matrix $A$ has the same weight, the approximate values in matrix $A$ are:

        \[
            A = \begin{bmatrix} 
                0.25 & 0.25 & 0.25 & 0.25 \\
                0.25 & 0.25 & 0.25 & 0.25 \\
                0.25 & 0.25 & 0.25 & 0.25
            \end{bmatrix}
        \]

    \subsubsection{Test to Refine Values for Matrix $A$}

        To refine these values, I will now consider how each output is affected by each input.  All have a positive correlation, and I will rate the correlation as strong, moderate, or weak.  I will then adjust the corresponding matrix values, keeping the sum of each matrix row as 1.

        \textbf{For difficulty ($d$):}
        \begin{itemize}
	        \item independence ($h$) – strong positive correlation (increase $a_{11}$ to $0.35$)
	        \item confidence ($c$) – moderate positive correlation (decrease $a_{12}$ to $0.15$)
	        \item perseverance ($p$) – moderate positive correlation (decrease $a_{13}$ to $0.15$)
	        \item accuracy ($m$) – strong positive correlation (increase $a_{14}$ to $0.35$)
        \end{itemize}

        \textbf{For support ($s$):}
        \begin{itemize}
	        \item independence ($h$) – strong positive correlation (increase $a_{21}$ to $0.30$)
	        \item confidence ($c$) – strong positive correlation (increase $a_{22}$ to $0.30$)
	        \item perseverance ($p$) – strong positive correlation (increase $a_{23}$ to $0.30$)
	        \item accuracy ($m$) – moderate positive correlation (decrease $a_{24}$ to $0.10$)
        \end{itemize}

        \textbf{For pacing ($t$):}
        \begin{itemize}
	        \item independence ($h$) – moderate positive correlation (increase $a_{31}$ to $0.30$)
	        \item confidence ($c$) – weak positive correlation (decrease $a_{32}$ to $0.15$)
	        \item perseverance ($p$) – weak positive correlation (decrease $a_{33}$ to $0.15$)
	        \item accuracy ($m$) – strong positive correlation (increase $a_{34}$ to $0.40$)
        \end{itemize}

        \textbf{Updated adjusted matrix $A$:}

        \[
            A = \begin{bmatrix} 
                0.35 & 0.15 & 0.15 & 0.35 \\
                0.30 & 0.30 & 0.30 & 0.10 \\
                0.30 & 0.15 & 0.15 & 0.40
            \end{bmatrix}
        \]
 
\end{document}